\documentclass[times, utf8, zavrsni]{fer}
\usepackage{booktabs}

\begin{document}

% TODO: Navedite broj rada.
\thesisnumber{920}

% TODO: Navedite naslov rada.
\title{Web-aplikacija za praćenje podataka o nogometnim utakmicama.}

% TODO: Navedite vaše ime i prezime.
\author{Marko Jerkić}

\maketitle

% Ispis stranice s napomenom o umetanju izvornika rada. Uklonite naredbu \izvornik ako želite izbaciti tu stranicu.
% TODO: Treba li mi izbornik?
% \izvornik

% Dodavanje zahvale ili prazne stranice. Ako ne želite dodati zahvalu, naredbu ostavite radi prazne stranice.
\zahvala{}

\tableofcontents

\chapter{Uvod}
Uvod rada. Nakon uvoda dolaze poglavlja u kojima se obrađuje tema.
Test uvod.

\chapter{Analiza postojećih rješenja na internetu}
Za analizu postojećih rješenja na internetu su korištene aplikacije Sofascore\footnote{https://sofascore.com}, OneFootball\footnote{https://onefootball.com} i Rezultati.com\footnote{https://rezultati.com}.

\section{Prikaz liste utakmica}

Jedan od glavnih dijelova aplikacija koje se koriste za praćenje nogometnih rezultata je stranica pregleda popisa dostupnih utakmica.
Većina postojećih rješenja na internetu koriste ovu stranicu kao središnju lokaciju s koje se pregledava većina sadržaja. To je istina za aplikacije Sofascore i Rezultati.com.

U slučaju te dvije aplikacije, prilikom pristizanja na samu web-lokaciju, prikazana nam je liste postojećih utakmica koje se igraju da dan posjećivanja aplikacije.

Središnja komponenta početne stranice Sofascore-a je lista utakmica grupirana po natjecanjima i državama. Prvih nekoliko prikazanih natjecanja su istaknuta ili spremljena natjecanja.
U slučaju kada stranicu posjećuje korisnik koji prvi put posjećuje stranicu ili koji nema spremljenih podataka u pregledniku, istaknuta natjecanja uključuju velika svjetska natjecanja,
kao engleska Premier liga, njemačka Bundesliga i slično te natjecanja koja se smatraju velikim u državi iz koje korisnik posjećuje stranicu.
Za korisnika koji posjećuje stranicu iz Hrvatske, među istaknutim natjecanjima se nalazi Hrvatska nogometna liga.
Nakon istaknutih natjecanja, prikazuju se natjecanja iz cijelog svijeta koja sadrže utakmice za odabrani dan. Natjecanja su grupirana po državama u kojima se održava to natjecanje.

Osim popisa utakmica, koji se nalazi u sredini aplikacije, Sofascore sadrži kalendar s lijeve strane popisa utakmica.
Kalendar se koristi za odabir datuma prema kojemu prikazuju dostupne utakmice.
Ispod kalendara se nalazi još popis istaknutih natjecanja i popis svih dostupnih natjecanja s poljem za sužavanje za unos teksta koji se koristi za sužavanje popisa natjecanja.

S desne strance popisa utakmica se nalazi pretpregled istaknute utakmice. Pretpregled sadrži ime i grb ekipa koje su sudjelovale u utakmici te rezultat utakmice.
Ispod toga se nalazi popis istaknutih igrača te reklame za sportsko klađenje.

Aplikacija Rezultati.com ima dosta sličan raspored na početnoj stranici aplikacije, ali ne prikazuje istaknuta natjecanja, utakmice niti igrače.
Aplikacije OneFootball ne koristi popis dostupnih utakmica na početnoj stranici. OneFootball je aplikacija koja se fokusira prvenstveno na nogometne novosti.
Do stranice na kojoj se prikazuje popis dostupnih utakmica se dolazi preko poveznice koja se nalazi u navigacijskoj traci.

\section{Prikaz detalja utakmice}

Sve tri aplikacije koje su korištene za analizu postojećih riješenja imaju dosta različito ponašanje prilikom odabira utamice s popisa.
Prilikom odabira utakmice na aplikaciji Sofascore otvori se pretpregled desno od popisa utakmica.
Aplikacija Rezultati.com otvori detalje utakmice u skočnom prozoru, dok OneFootball otvori novu stranicu.

Zaglavlje stranice detalja utakmice prikazuje ime i grb ekipa koje su sudjelovale u utakmici, datum utakmice i trenutni rezultat.
Detaljni podatci su raspodijeljeni u više kategorija koje se odabiru preko kartica. Zadana kartica sadrži događaje utakmice, kao što su pogodci, kartoni i zamjene, kronološki poredane po minuti.

Pored kartice događaja po minuti, bitne su još kartice početnih postava dviju ekipa te kartica statistike utakmice.
Kartica početnih postava utakmice grafički prikazuje prvih jedanaest igrača obije ekipe, formaciju, boju dresova te broj na dresovima.
Kartica statistike prikazuje podatke o posjedu lopte, broju udaraca, broju udaraca u vrata, broju udaraca iz kuta i sl., za obije ekipe.

\section{Prikaz detalja igrača, izbornika, ekipa i natjecanja}

Nakon stranica prikaza popisa utakmica i detalja pojedinačne utakmice, sporednu ulogu u odabranim aplikacijama zauzimaju stranice pregleda detalja igrača, izbornika, ekipa i natjecanja.

Stranice pregleda detalja igrača i izbornika su dosta slične. Obije prikazuju ime osobe, sliku, datum rođenja i nacionalnost.
Ispod toga se prikazuju kartice s popisom najnovijih utakmica u kojima je sudjelovala ta osoba, osobna statistika te popis ekipa za koje je ta osoba igrala ili koje je ta osoba trenirala.

Stranica prikaza detalja o ekipi prikazuje ime ekipe, grb i državu iz koje dolazi ekipa. Nakon toga se prikazuju kartice s podatcima o nedavnim utakmicama koje je igrala ekipa,
trenutnim igračima koji igraju za tu ekipu, te tablicu poretka natjecanja u kojemu ta ekipa sudjeluje.

Kod stranice pregleda natjecanja, najbitnije su kartice s tablicom poretka te popisom utakmica po sezoni za odabrano natjecanje.

\chapter{Zaključak}
Zaključak.

\bibliography{literatura}
\bibliographystyle{fer}

\begin{sazetak}
Ukratko opisati najznačajnije mogućnosti odabranih postojećih aplikacija namijenjenih prikazu statističkih
podataka o nogometnim utakmicama. Oblikovati relacijski model baze podataka za praćenje relevantnih
podataka o klubovima, igračima, sudjelovanju igrača u utakmicama, formacijama, rangu natjecanja,
zamjenama, trenerima ili izbornicima, posjedu lopte, asistencijama, udarcima u vrata i slično. Na osnovi
izgrađenog modela implementirati bazu podataka i web-aplikaciju za prikupljanje podataka i prikaz obrađenih
podataka o utakmicama, klubovima i igračima u prikladnom korisničkom sučelju. Za odabrani skup prikazanih
podataka osigurati osvježavanje podataka u korisničkom sučelju u stvarnom vremenu.

\kljucnerijeci{Ključne riječi, odvojene zarezima.}
\end{sazetak}

% TODO: Navedite naslov na engleskom jeziku.
\engtitle{Title}
\begin{abstract}
Abstract.

\keywords{Keywords.}
\end{abstract}

\end{document}
