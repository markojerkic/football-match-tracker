\documentclass[times, utf8, zavrsni]{fer}
\usepackage{booktabs}

\begin{document}

% TODO: Navedite broj rada.
\thesisnumber{920}

% TODO: Navedite naslov rada.
\title{Web-aplikacija za praćenje podataka o nogometnim utakmicama.}

% TODO: Navedite vaše ime i prezime.
\author{Marko Jerkić}

\maketitle

% Ispis stranice s napomenom o umetanju izvornika rada. Uklonite naredbu \izvornik ako želite izbaciti tu stranicu.
% TODO: Treba li mi izbornik?
% \izvornik

% Dodavanje zahvale ili prazne stranice. Ako ne želite dodati zahvalu, naredbu ostavite radi prazne stranice.
\zahvala{}

\tableofcontents

\chapter{Uvod}
Uvod rada. Nakon uvoda dolaze poglavlja u kojima se obrađuje tema.
Test uvod.

\chapter{Zaključak}
Zaključak.

\bibliography{literatura}
\bibliographystyle{fer}

\begin{sazetak}
Ukratko opisati najznačajnije mogućnosti odabranih postojećih aplikacija namijenjenih prikazu statističkih
podataka o nogometnim utakmicama. Oblikovati relacijski model baze podataka za praćenje relevantnih
podataka o klubovima, igračima, sudjelovanju igrača u utakmicama, formacijama, rangu natjecanja,
zamjenama, trenerima ili izbornicima, posjedu lopte, asistencijama, udarcima u vrata i slično. Na osnovi
izgrađenog modela implementirati bazu podataka i web-aplikaciju za prikupljanje podataka i prikaz obrađenih
podataka o utakmicama, klubovima i igračima u prikladnom korisničkom sučelju. Za odabrani skup prikazanih
podataka osigurati osvježavanje podataka u korisničkom sučelju u stvarnom vremenu.

\kljucnerijeci{Ključne riječi, odvojene zarezima.}
\end{sazetak}

% TODO: Navedite naslov na engleskom jeziku.
\engtitle{Title}
\begin{abstract}
Abstract.

\keywords{Keywords.}
\end{abstract}

\end{document}
